\documentclass{article}
\usepackage[utf8]{inputenc}
\setlength{\parindent}{0cm}
\addtolength{\hoffset}{-2cm}
\addtolength{\textwidth}{4cm}
\usepackage[frenchb]{babel}
\usepackage[T1]{fontenc}
\usepackage{hyperref}
\usepackage{graphicx}   
\usepackage{bibtex}   

\title{DotAI - Projet de semestre}
\author{Thomas Ibanez}


\begin{document}
\maketitle

\section{Introduction}

Le but de ce projet de semestre est de créer un ensemble de logiciels permettant de présenter les données contenues sur \url{opendota.com} pour le développement d'un logiciel d'apprentissage automatique sur le jeu DotA 2

\subsection{DotA 2}

DotA 2 est jeu-vidéo multijoueur de type bataille en arène (MOBA) où 2 équipes s'affrontent. Chaque équipe est composée de 5 joueurs chacun controllant un héro choisi parmis les héros disponnibles dans le jeu (il ne s'agit donc pas d'un avatar personnalisable comme dans le cas des MMORPG). Il y a à l'heure actuelle le jeu propose 115\bibitem{bl}115, \url{http://www.dota2.com/heroes/?l=french} héros, chacuns disposant de 5 attaques différentes La partie se déroule sur une carte symétrique dont voici une vue aérienne: \\
\includegraphics[scale=1]{"minimap.png"}



\end{document}
